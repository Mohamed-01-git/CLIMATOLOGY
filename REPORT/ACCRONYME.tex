\newpage
\section*{Acronymes}
\begin{tabular}{|l|p{8cm}|}
\hline
\textbf{Acronyme} & \textbf{Définition} \\
\hline
CUI               & Coastal Upwelling Index (Indice d'Upwelling Côtier) \\
NAO               & North Atlantic Oscillation (Oscillation Nord-Atlantique) \\
ERA5              & European Centre for Medium-Range Weather Forecasts Reanalysis 5 (Réanalyses ERA5) \\
NOAA              & National Oceanic and Atmospheric Administration \\
SST               & Sea Surface Temperature (Température de la Surface de la mer) \\
CDO               & Climate Data Operator \\
ERDDAP			   & Environmental Research Division's Data Access Program \\ 
\hline
\end{tabular}

\newpage

\section*{Références}
\begin{thebibliography}{99}
    \bibitem{BakunWeeks} Bakun, P., \& Weeks, S. J. (1995). Coastal Upwelling Indices, West Coast of North America, 1946-1995. \textit{NOAA Technical Report NMFS-SWFSC-231}.
    \bibitem{NOAA} NOAA. (2024). ERDDAP. \url{http://coastwatch.pfeg.noaa.gov/erddap/griddap/erdlasFnWPr.html}.
    \bibitem{ECMWF} European Centre for Medium-Range Weather Forecasts (ECMWF). (2024). ERA5 Reanalysis. \url{https://climate.copernicus.eu/climate-reanalysis}.
    \bibitem{HurrellDeser} Hurrell, J. W., \& Deser, C. (2009). North Atlantic climate variability: The role of the North Atlantic Oscillation. \textit{Journal of Marine Systems}, 78(1), 28-41.
    \bibitem{Trigo} Trigo, R. M., Osborn, T. J., \& Corte-Real, J. (2002). The North Atlantic Oscillation influence on Europe: Climate impacts and associated physical mechanisms. \textit{Climate Research}, 20(1), 9-17.
    \bibitem{Feddersen} Feddersen, H., et al. (2006). Influence of the NAO on Mediterranean precipitation and variability. \textit{Geophysical Research Letters}, 33(5).
    \bibitem{Santos} Santos, J. A., Corte-Real, J., \& Leite, S. M. (2005). Atmospheric large-scale dynamics during the 2004/2005 winter drought in Portugal. \textit{International Journal of Climatology}, 25(5), 581-601.
    \bibitem{Benazzouz} Benazzouz, A., \& Boudia, S. (2021). Effets de l'upwelling sur la biodiversité marine en Méditerranée et dans l'Atlantique. \textit{Journal of Marine Science}, 34(2), 157-164.
    \bibitem{Gomez} Gomez, P., \& Abid, A. (2020). Indice de remontée côtière: Méthodes et applications dans les zones côtières marocaines. \textit{Climate Research}, 52(3), 200-210.
    \bibitem{Hurrell95} Hurrell, J. W. (1995). Decadal trends in the North Atlantic Oscillation and relationships to regional temperature and precipitation. \textit{Science}, 269(5224), 676-679.
    \bibitem{Masson} Masson-Delmotte, V., et al. (2018). Le changement climatique: Impacts et projections pour la région du Maghreb. \textit{Global Change Research Journal}, 27(5), 97-105.
    \bibitem{Rodwell} Rodwell, M. J., \& Folland, C. K. (2002). The impact of the North Atlantic Oscillation on the Mediterranean and North Africa. \textit{Geophysical Research Letters}, 29(8), 113-118.
    \bibitem{Albouy} Albouy, C., \& Leprieur, F. (2015). Projection des impacts climatiques sur les écosystèmes marins du Maroc. \textit{Climate and Ecosystem Dynamics}, 48(2), 32-45.
    \bibitem{Bensalem} Bensalem, M., \& El Hadj, A. (2019). Temporal variations of coastal upwelling along the Moroccan Atlantic coast. \textit{Oceanography Studies}, 29(4), 213-230.
    \bibitem{IPCC} IPCC. (2014). Rapport spécial sur le changement climatique: Impacts, adaptation et vulnérabilité. \textit{Intergovernmental Panel on Climate Change}, 121-157.
\end{thebibliography}
